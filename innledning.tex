\chapter{Innledning}
\section{Bakgrunn}
Serviceenheten IKT ved Hedmark Fylkeskommune (heretter kalt IKT-avdelingen) har det overordnede ansvaret for alt datautstyr som tilhører fylkeskommunen, og datanettverket for sentraladministrasjonen og øvrige lokasjoner. Dette omfatter blant annet videregående skoler, tannklinikker og Hedmark Trafikk. Totalt benyttes IT-løsningene av over 10 000 brukere.

Datasystemene er kritiske for daglig drift av de mange funksjonene fylkeskommunen fyller. Det er derfor mange brukere som berøres dersom en feil skulle inntreffe i systemene. Det er det viktig at de ansvarlige får beskjed så raskt og effektivt som mulig når et system går ned, eller ikke fungerer som det skal.

Per i dag benyttes en enkel overvåkningsløsning, ``Servers Alive''\cite{servers}, som ikke dekker IKT-avdelingens behov. Oppsettet gir for mye irrelevant informasjon, som gjør at en ikke får et godt nok oversiktsbilde. Det mangler også mulighet for å kunne overvåke mange ønskede tjenester. Systemet gir kun varsling til skjerm, men det er i tillegg ønskelig å kunne få varsler på SMS og e-post. Servers Alive er også en proprietær løsning, og dermed ikke lett å utvide. I tillegg til dette kjøres ikke sjekkene parallelt, og derfor skalerer systemet dårlig.

IKT-avdelingen har lenge arbeidet med å finne en bedre overvåkningsløsning. Det er ønskelig å erstatte Servers Alive med en mer modulbasert løsning, der IKT-avdelingen har tilgang til kildekoden og kan utføre endringer. De har imidlertid for hver gjennomgang av mulige overvåkningsløsninger konkludert med at de ikke har de nødvendige ressursene til et slikt prosjekt. Da vi forespurte IKT-avdelingen om de hadde noen mulige bachelorprosjekter, var de veldig interessert i at en bachelorgruppe kunne se på hvordan et slikt prosjekt kunne gjennomføres.

\section{Oppgavebeskrivelse}
Det skal settes opp en overvåkningsløsning, som skal varsle feil og hendelser på enheter i fylkeskommunens datanettverk. Dette innebeærer servere med tjenester, og nettverksenheter som switcher, routere og brannmurer. Dette systemet må være modulært med mulighet for enkelt å kunne legge til nye moduler i ettertid. Følgende krav er satt til løsningen:

\begin{itemize*}
	\item Webgrensesnitt for administering av overvåkningssystemet, der informasjon om enheter og hvilke tjenester som overvåkes skal vises.
	\item Varsling
	\begin{itemize*} 
		\item med feilbeskrivelse.
		\item med flere grader kritisk nivå.
		\item med mulighet for varsling til SMS, e-post og skjerm.
		\item som tar hensyn til redundans.
	\end{itemize*}
	\item Avhengigheter mellom enheter som overvåkes skal kunne settes opp. Varsling skal gi informasjon om det er antatt følgefeil og hovedfeil.
	\item Objekter (data inn) og data skapt av programvaren (data ut) skal lagres i database(r).
	\item All tiltstandsinformasjon skal logges og lagres i database.
	\item WMI, SNMP, ICMP (med antall drop før varsling), og eventuelt andre aktuelle standarder skal støttes.
	\item Det skal være enkelt å legge til nye enheter for overvåking, fra et grafisk grensesnitt.
	\item Innen servermiljø skal det kunne overvåkes temperatur, luftfuktighet og UPS.
\end{itemize*}

Vi må derfor finne programvare som kan tilpasses fylkeskommunens systemer, og som dekker flest mulig av disse kravene. Dersom eksisterende løsninger ikke finnes, eller ikke dekker behovet, må vi utvikle dette selv.

IKT-avdelingen har også en liste over tilleggsfunksjonalitet de ønsker implementert i løsningen. Disse vil implementeres dersom det er tid til det.

\begin{itemize*}
	\item Rapporter som viser dagens situasjon og trender.
	\item Ta hensyn til driftsarbeid. Planlagt arbeid definert på enheter vil ikke utløse alarm, men driftsarbeidsvarsling.
	\item SLA
	\begin{itemize*}
		\item Vise SLA-avtaler opp mot faktiske verdier.
		\item Kundebase med krav opp mot logget data.
		\item Kapasitetsmålinger opp mot maksverdier (for eksempel linjekapasitet).
	\end{itemize*}
	\item Dashboard-funksjonalitet med mulighet for personliggjøring av grafisk grensesnitt.
	\item Grensesnitt for mobile enheter.
	\item Konfigurasjonskart som tegner kart av løsning grafisk med klikkbare objekter (enheter).
	\item Mulighet for lagring av konfigurasjonsoppsett for objekter med integrasjon mot CMDB.
	\item Overvåking av telefonlinjer via Trio og Asterisk.
\end{itemize*}

Etter at valg av kjerneprogramvare var tatt ble det definert at følgende kategorier med underpunkter skulle overvåkes:
\begin{itemize*}
	\item Infrastruktur og nettverk.
	\begin{itemize*}
		\item Brannmurer.
		\item Switcher.
		\item Trådløse kontrollere.
		\item VMware ESX og Citrix Xen.
		\item Serverrommiljø.
		\begin{itemize*}
			\item Temperatur.
			\item Strøm (UPS).
			\item Luftfuktighet.
		\end{itemize*}
	\end{itemize*}
	\item Servere
	\begin{itemize*}
		\item CPU.
		\item Minne.
		\item Disk.
		\item Prosesser.
		\item Applikasjoner.
		\item Databaser: MySQL, MySQL Cluster, Oracle og MSSQL.
		\item E-postserver: Microsoft Exchange.
		\item Tjenester.
		\begin{itemize*}
			\item Webservere.
			\item DNS.
			\item DHCP.
			\item LDAP.
		\end{itemize*}
	\end{itemize*}
\end{itemize*}

\section{Avgrensninger}
Fordi oppdragsgiver har både krav og ønsket tilleggsfunksjonalitet, må oppgaven avgrenses slik at tidsrammer for å oppfyllene kravene overholdes. Dersom vi ligger foran tidsskjemaet, vil vi begynne å se på og implementere tilleggsfunksjonalitet.

Manuell endring av konfigurasjon i tekstfiler er den mest fleksible løsningen, og oppdragsgiver var enig i at dette var enkelt nok og dermed ikke trengte å være grafisk.

Kravet til at alle data inn og ut skal logges i databaser bortfaller, da dette gjennom testing genererte for store datamengder til at det var hensiktsmessig å lagre alt.

\section{Målgruppe}
Overvåkningsløsningen utvikles for IKT-avdelingen, og ansatte her vil være en av målgruppene. Selve rapporten vil vinkles slik at medstudenter, veileder, oppdragsgiver, sensor og ellers andre interesserte kan få et innblikk i hvordan prosessen med å implementere en overvåkningsløsning basert på fri programvare har vært. Det faglige nivået på rapporten vil være slik at medstudenter kan få utbytte av stoffet.

\section{Prosjektmål}
\subsubsection{Effektmål}
I forhold til dagens løsning skal den nye løsningen
\begin{itemize*}
	\item være mer oversiktlig.
	\item tilrettelegge for mer omfattende overvåkning.
	\item være mer fleksibel.
	\item bidra til mer effektiv drift for IKT-avdelingen, der hendelser raskt kan oppdages og eskaleres.
\end{itemize*}

\subsubsection{Resultatmål}
\begin{itemize*}
	\item Utvikle en overvåkningsløsning som tilfredsstiller de krav som er satt.
	\item Den nye overvåkningsløsningen vil resultere i at den eksisterende kan erstattes og deretter fases ut.
\end{itemize*}

\subsubsection{Læringsmål}
\begin{itemize*}
	\item Sette oss inn i ``best practices'' for overvåking av datasystemer.
	\item Kunne tilpasse og implementere en overvåkningsløsning for små og mellomstore bedrifter.
\end{itemize*}

\section{Rammer}
Følgende krav var gitt i oppgaven
\begin{itemize*}
	\item Dersom eksisterende programvare benyttes må denne være fri programvare.
	\item MySQL skal benyttes som databasesystem.
	\item Prosjektrapporten skal ikke inneholde sensitiv informasjon om Hedmark fylkeskommunes tekniske løsninger. Eksempelvis IP-adresser, brukernavn/passord eller brannmurkonfigurasjon. Dette er i tråd med taushetserklæringen som må inngås.
	\item Det skal kjøpes inn en enhet for å overvåke luftfuktighet og temperaturer. Innkjøp av denne skal skje i henhold til fylkeskommunens innkjøpsrutiner.
\end{itemize*}

\section{Faglig bakgrunn}
Bjørn-Erik og Øyvind er 3. års studenter på Drift av nettverk og datasystemer, mens Nils går 3. året på Dataingeniør. Gjennom to og et halvt år på Høgskolen i Gjøvik har vi tatt fag innenfor områdene programmering, nettverk, databaser, sikkerhet, statistikk og ulik grad av matematikk. Nils og Øyvind har tidligere vært lærlinger ved henholdsvis Hedmark fylkeskommune og Hamar katedralskole, der de fikk kunnskap om deres applikasjoner og infrastruktur.

Bjørn-Erik og Øyvind har også tidligere skrevet en rapport om bruk av ITIL ved Hedmark fylkeskommunes IKT-avdeling, i faget ``IT Service Management''.

Vi mener at en overvåkningsoppgave med denne beskrivelsen kan dekke mange av områdene vi har vært gjennom i løpet av studiet. Her er det også områder der gruppen har delvis eller manglende kunnskap, både om hvilke teknologier som finnes og hvordan de brukes i et større datamiljø.

\section{Roller}
\subsubsection{Prosjektleder}
Prosjektleder er Øyvind Sigerstad. Denne rollen vil være fast under hele prosjektet. Lederens hovedansvar er å se nødvendigheten for møter, samt ha et overordnet ansvar for arbeidsfordelingen. Prosjektlederen vil også ta endelige avgjørelser dersom gruppen ikke kan komme til enighet, som stipulert i gruppereglene.

\subsubsection{Kommunikasjonsansvarlig}
Vår kontaktperson er Nils Slåen. All kontakt med oppdragsgiver og annen utgående kommunikasjon vil gå via ham. Kommunikasjonsansvarlig har også ansvar for å avtale møter.

\subsubsection{Webansvarlig}
Vår webansvarlig er Bjørn-Erik Strand. Hans hovedansvar er å sette websiden vår i drift før fristen. Webansvarlig har også et overordnet ansvar for å publisere oppdateringer. De andre medlemmene i gruppen skal også ha tilgang til å legge ut informasjon.

\subsubsection{Oppdragsgiver}
Vår oppdragsgiver er Svein-Inge Kvalø (Fungerende driftsleder ved IKT-avdelingen, Hedmark fylkeskommune), og er vårt kontaktpunkt ved Hedmark fylkeskommune. Lasse Odden er vår tekniske kontakt (IT-konsulent ved IKT-avdelingen, Hedmark fylkeskommune). Han vil være en ressurs som brukes for tekniske spørmål rundt IKT-avdelingens oppsett.

\subsubsection{Veileder}
Erik Hjelmås (Førsteamanuensis, Dr. scient ved Høgskolen i Gjøvik), er vår veileder gjennom bacheloroppgaven. Han vil bidra til avklaringer rundt oppgaven og komme med innspill rundt tekniske utfordringer.

\section{Rutiner og regler i gruppa}
Et eget dokument er utarbeidet med grupperegler som alle gruppens medlemmer er enige i og har skrevet under på. Dette finnes i Vedlegg \ref{app:forprosjekt} side 14.

\section{Verktøy}
Alt av egenutviklet kildekode og konfigurasjonsfiler vil bli lagt under et SVN repository. Dette for å få versjonskontroll og samle alt på et sted, som vi kan ta backup av. Google Docs vil bli brukt for å samarbeide på dokumenter. Når den endelige rapporten skal genereres, vil vi overføre Google Docs-dokumentet til \LaTeX. Dropbox vil bli benyttet til å dele filer og dokumenter innad i prosjektgruppen, og med oppdragsgiver. Vi vil bruke Wordpress som publiseringsverktøy for hjemmesiden.

Av andre verktøy skal Microsoft Visio benyttes til å lage illustrasjoner og matplotlib til diagrammer. 

\section{Utviklingsmodell/rammeverk}
For å tilpasse overvåkningssystemet til IKT-avdelingens krav, vil det med stor sannsynlighet komme endringer etterhvert. Det vil kunne oppstå ulike situasjoner som er vanskelige å forutse. Eksempler på dette er at en plugin som kreves ikke eksisterer, eller at webgrensesnittet må endres på grunn av mer informasjon eller omorganisering. I tillegg kan oppdragsgiver komme med innspill som fører til endring av funksjonalitet. Derfor må vi ha en fleksibel modell med tanke på endringer og tilbakefall, men med begrensninger på hvor mye tid vi kan bruke ved uforutsette problemer.

På grunn av en naturlig sammenheng og avhengigheter mellom kravene som er satt, har vi valgt å gruppere funksjonalitet i moduler. De forskjellige modulene er ``Kjerneprogramvare'', ``Server'', ``Infrastruktur'', ''Servermiljø'', ``Varsling'' og ``Statusvindu'' (visualisert i Vedlegg \ref{app:forprosjekt} side 12). Med nærliggende sammenheng mener vi at funksjonalitet som CPU, RAM, disk og prosesser er i samme kategori. 

Planen er også å gjennomføre kategoriene i en gitt rekkefølge. Varsling kan settes opp uten noe input, men med tanke på redundans i systemene, avhengigheter mellom objektene, og testing av funksjonene som går inn under varsling, ser vi på dette som en avhengighet av Server og Infrastruktur. Statusvindu har vi valgt å legge sist fordi vi ser for oss at modulene Server, Infrastruktur, Servermiljø, og Varsling må være på plass før en varslingsskjerm kan ferdigstilles.

I tradisjonell systemutvikling må en definere alle parametere og rekkefølge på alt som skal utvikles, før en starter med selve utviklingen. En smidig modell vil gi oss mer frihet til å gjøre justeringer underveis. Vi har derfor valgt en iterativ modell\cite{wiki:iterativ}. En del av den iterative modellen er ``product control list'', men denne har vi valgt å ikke ta med fordi vi har satt opp en overordnet kategorisering av funksjonalitet, og bestemt når hver av disse skal implementeres.

Vi vil dele inn iterasjonene i faser, som hver består av en planleggingsfase, en implementeringsfase, og en fase for testing og evaluering. Disse vil så gjentas for hver modul som skal implementeres. Halvveis i hver iterasjon vil vi ha et møte med oppdragsgiver der vi kan presentere funksjonalitet i modulen, og få tilbakemeldinger. Det gjør at vi kan jobbe med dette i den siste halvdelen av iterasjonen. Dette vil forhåpentligvis føre til at produktet ikke sporer av fra de forventningene og behovene IKT-avdelingen har.

\section{Om rapporten}
\subsection{Oversikt over kapitler}
\begin{enumerate*}
\item Innleding - oppgaven inntroduseres. Bakgrunn, formål og plan for gjennomføring gjennomgås.
\item Teori - en teoretisk bakgrunn innen overvåkning og konfigurasjon av overvåkningsløsningen.
\item Implementasjon - gjennomgang av implementasjonen av overvåkningsløsningen.
\item Sikkerhet - de sikkerhetsvurderingene gruppen har tatt i implementasjonen presenteres.
\item Dokumentasjon - rutiner og informasjon gruppen anser som nyttig tilleggsinformasjon til implementasjonen.
\item Avslutning - rapporten avsluttes og konkluderes.
\end{enumerate*}
\subsection{Konvensjoner}
Kildekode vil inneholde ``syntax-highlighting'' til riktig programmeringsspråk og linjenummer. Dette markeres slik:
\begin{lstlisting}[language=perl]
 #!/usr/bin/perl
 print "Hello World.\n";
\end{lstlisting}

Konfigurasjonsfiler og kommandoer markeres slik:
\begin{lstlisting}[style=example]
define command {
   command_name   check_ssh
   command_line   $USER1$/check_ssh -H $HOSTADDRESS$ -p $_HOSTSSHPORT$
}
\end{lstlisting}
En linje i en konfigurasjonsfil omtales som et ``direktiv''. En spesifikk verdi omtales som et ``attributt'' (for eksempel command\_name).
Der kun utdrag av konfigurasjon eller kildekode er benyttet brukes ``...'' for å markere hvor det er utelatt.

Objekttyper omtegnes med det navnet som er brukt i dokumentasjonen for eksempel ``host'', ``service'', ``timeperiod''. ``Tjeneste'' brukes om applikasjoner påpå en server, som benyttes av andre enheter og/eller brukere. ``Host'' eller ``enhet'' brukes om en enhet med en IP-adresse. 
\subsection{Grafikk}
Grafene i rapporten er generert av Graphite. Alle illustrasjoner er enten egenprodusert eller hentet fra Icingas dokumentasjon, der de er publisert under GPLv2.
