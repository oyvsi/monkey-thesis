\chapter*{Forord}
Da forslagene til bacheloroppgaver ble presentert, var det ingen oppgaver vi følte utpekte seg som ekstra interessante. Vi ønsket oss en oppgave der vi kunne ha en implementasjon- eller utviklingsdel, der vi fikk bruk for mest mulig av den kunnskapen vi har opparbeidet oss gjennom to og et halvt år ved Høgskolen i Gjøvik. 

Vi tok da kontakt med IKT-avdelingen ved Hedmark fylkeskommune for å høre om de hadde noen mulige problemstillinger som kunne passe som en bacheloroppgave. Dette resulterte i en oppgave som passet med våre ønsker, og som vi syntes var meget interessant. Det skulle utvikles et forslag til en ny overvåkningsløsning av servere, infrastruktur og tjenester. 

Denne rapporten dekker hvordan vi har gått fram for å fylle flest mulig av avdelingens behov, og hvilke valg og vurderinger vi har tatt underveis.
 
Gruppen takker:
\begin{itemize} 
\item Veileder Erik Hjelmås for god hjelp og veiledning gjennom hele prosjektet.
\item Svein-Inge Kvalø, Lasse Odden og resten av IKT-avdelingen ved Hedmark fylkeskommune for god kommunikasjon, samarbeid og tilrettelegging for oppgaven.
\item Bachelorgruppen dot11ac for bilberging og korrekturlesing.
\item Jon Langseth for gjennomgang av overvåkningsløsningen som IT-tjenesten ved HiG bruker.
\end{itemize}
