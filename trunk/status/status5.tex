\documentclass[11pt,norsk,a4paper]{article}

\usepackage[utf8]{inputenc}     % For utf8 encoded .tex files
\usepackage[absolute]{textpos}
\usepackage[T1]{fontenc}
\usepackage{babel,graphicx,mathpple,textcomp,varioref,pdfpages}           % For inclusion of graphics
\usepackage{geometry}

\addtolength{\topmargin}{-1in}
\pagenumbering{gobble}
\title{MonKey: Statusrapport}


\date{Fredag 29. Mars 2013}

\begin{document}

\maketitle
\section*{Femte iterasjon: Servermiljø}

\noindent I denne iterasjonen har fokus vært på miljøet inne på serverrommet. Her har bestilt utstyr blitt montert i rack på fylkeshuset. Ledninger har blitt trukket fra overvåknings sentralen og til de fire temperatur- og luftfuktighetsfølerne. To av følerne er montert på forsiden der luften suges igjennom de rackmonterte serverne, og to bak for å se hvor mye temperaturer og luftfuktighet påvirkes av dette.\\

\noindent Selve sentralen er en APC netbotz 200. Denne er modulerbar og flere sensorer kan kobles til dersom man skulle ønske dette. I Icinga overvåkes nå to tjenester på sentralen, dette er luftfuktighet og temperaturer for alle følerne. Vi samler også inn performance data på dette slik at det kan lages grafer for å finne ut av trender osv for disse to måleenhetene.\\

\noindent Etter statusmøtet med Lasse har vi fått gode tilbakemeldinger at dette var slik de så for seg at serverrommet ble overvåket, samt at dette kommer til å bidra til å samle trender og gjøre miljøet bedre og sikkrere ved hjelp av trend data.\\

\noindent Siden denne iterasjonen har foregått gjennom påskeferien har vi valgt å utsette statusrapporten før hele gruppen, Svein-Inge og Lasse har hatt et statusmøte.\\

\noindent I denne iterasjonen skulle vi også ha lagt inn overvåkning for UPS, siden vi alt har gjort mesteparten av dette i infrastruktur modulen gjenstod det her bare å gå igjennom å kvalitetsikkre tresholds og performance data som var samlet inn.\\

\noindent Parallelt med denne iterasjonen vil vi jobbe med statusvisning, statusrapport for denne kommer i en egen rapport.\\

\end{document}
